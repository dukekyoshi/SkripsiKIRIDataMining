\chapter{Pendahuluan}
\label{chap:intro}

\section{Latar Belakang}
\label{sec:motivation}

KIRI merupakan suatu organisasi yang memiliki tujuan untuk mengurangi pemanasan global, kemacetan, dan harga BBM tinggi. Untuk mencapai hal tersebut, KIRI memberikan solusi dengan mengajak Indonesia untuk menggunakan fasilitas transportasi publik. Maka dari itu, KIRI bergerak di bidang transportasi publik dengan memberikan petunjuk jalan menggunakan transportasi publik yang dapat diakses melalui website atau handphone. Hingga saat ini, KIRI sudah menyentuh 3 wilayah, yaitu Bandung, Jakarta, dan Surabaya.

Pertumbuhan teknologi hingga saat ini telah menghasilkan banyak sekali data-data, namun sering kali pemilik data hanya menggunakan data tersebut seperlunya saja. Jika dilihat lebih rinci, sebenarnya jika data tersebut diolah lebih lanjut, dapat menghasilkan sesuatu yang lebih. Perusahaan KIRI, telah menghasilkan banyak data log histori setiap bulan namun masih belum dilakukan tindakan pengolahan data. Salah satu cara mengolah data tersebut adalah dengan menggunakan teknik \textsl{data mining}. Dengan menggunakan teknik \textsl{data mining} akan mempermudah menganalisa masalah, pengambilan kesimpulan, bahkan mempermudah konsumen dalam menggunakan jasa.

Dengan menggunakan proses \textsl{data mining}, data log histori KIRI yang berisi informasi seorang pengguna berangkat dari suatu lokasi ke lokasi tertentu, dimungkinkan dapat menghasilkan suatu \textsl{pattern} yang menarik dan berguna baik untuk KIRI ataupun pemerintah. 

\section{Perumusan Masalah}
Dengan mengacu pada uraian diskripsi diatas, maka permasalahan yang dibahas dan diteliti oleh penulis adalah
\begin{itemize}
	\item Bagaimana cara mengolah pola yang diperoleh dari \textsl{data log histori}  KIRI agar pola menjadi menarik dan bermakna?
	\item Bagaimana membuat perangkat lunak untuk melakukan data mining pada data log history?
\end{itemize}

\section{Tujuan}
Penelitian ini bertujuan untuk 
\begin{itemize}
	\item Mencari pola dan informasi yang menarik dari \textsl{log histori} KIRI
	\item Perangkat lunak dapat melakukan data mining dari \textsl{log histori} KIRI
\end{itemize}

\section{Batasan Masalah}
Penelitian \textsl{data mining} yang diatas akan ditentukan batasan masalah yang diteliti berupa : 
\begin{itemize}
	\item Penelitian ini dibatasi hanya pada permasalahan pada penerapan \textsl{data mining} pada \textsl{data log} KIRI
	\item Data log yang merupakan masukan akan dibatasi sebanyak 10000 buah data
\end{itemize}

\section{Metode Penelitian}
Berikut adalah Metode Penelitian yang digunakan : 
	\begin{itemize}
		\item Melakukan studi literatur tentang algoritma-algoritma yang berkaitan dengan pemrosesan \textsl{data mining}
		\item Melakukan penelitian \textsl{data mining} yang diterapkan pada \textsl{log} KIRI
		\item Merancang dan mengimplementasikan algoritma untuk \textsl{data mining}
		\item Mengimplementasikan pembangkit pola \textsl{data mining}
		\item Melakukan pengujian dan eksperimen
	\end{itemize}

\section{Sistematika Pembahasan}
Sitematika pembahasan dalam penelitian ini adalah:
Bab 1: Pendahuluan, berisi latar belakang dari penelitian ini, rumusan masalah yang timbul, tujuan yang ingin dicapain, ruang lingkup atau batasan masalah dari penelitian ini, serta metode penelitian yang akan digunakan dan sistematika pembahasan dari penelitian ini.
Bab 2: Landasan Teori, berisi dasar teori mengenai \textsl{data mining} dan \textsl{spatial and Spatiotemporal}
Bab 3:
Bab 4:
Bab 5: