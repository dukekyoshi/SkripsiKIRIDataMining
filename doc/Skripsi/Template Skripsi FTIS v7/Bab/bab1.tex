\chapter{Pendahuluan}
\label{chap:intro}

\section{Latar Belakang}
\label{sec:motivation}

\textsl{Data mining} merupakan teknik untuk mengolah \textsl{data}, mencari suatu pola, kemudian dievaluasi untuk mendapatkan \textsl{knowledge}. Pola yang dihasilkan oleh \textsl{data mining} dapat digunakan untuk mencari suatu hal yang menarik dan unik. Dari pola tersebutlah, akan didapatkan suatu informasi yang sangat bermakna yang biasa disebut sebagai \textsl{knowledge}.

Pada penelitian ini, akan dilakukan analisis data penggunaan KIRI dengan menggunakan teori-teori \textsl{data mining}. Pada penelitian ini, diharapkan dapat ditemukan suatu pola yang menarik dan menghasilkan suatu ide atau gagasan yang baru. Maka dari itu, penelitian ini akan membutuhkan banyak \textsl{data} penggunaan KIRI, yang dapat diperoleh dari \textsl{log} histori KIRI.

\textsl{Log} tersebut memiliki 5 \textsl{field} untuk setiap \textsl{entry} sebagai berikut:
\begin{itemize}
	\item statisticId, primary key dari entry
	\item verifier, mengidentifikasikan sumber dari pencarian ini
	\item \textsl{timestamp}, waktu ketika pengguna KIRI mencari rute angkot
	\item \textsl{type}, tipe log, untuk penelitian ini selalu berisi FINDROUTE
	\item additionalInfo, mencatat koordinat awal, koordinat akhir, dan banyak rute yang ditemukan pada pencarian ini
\end{itemize}

Dengan menggunakan \textsl{data log} histori pencarian rute angkot dari KIRI, diharapkan bisa diperoleh pola yang menarik. Pola tersebut memiliki kemungkinan untuk digunakan kembali sebagai informasi tambahan pada perangkat lunak navigasi lain seperti KIRI.

\section{Perumusan Masalah}
Dengan mengacu pada uraian diskripsi diatas, maka permasalahan yang dibahas dan diteliti oleh penulis adalah
\begin{itemize}
	\item Bagaimana cara mengolah pola yang diperoleh dari \textsl{data log histori}  KIRI agar pola menjadi menarik dan bermakna?
	\item Bagaimana membuat perangkat lunak untuk melakukan data mining pada data log history?
\end{itemize}

\section{Tujuan dan Manfaat}
\subsection{Tujuan}
Penelitian ini bertujuan untuk 
\begin{itemize}
	\item Mencari pola dan informasi yang menarik dari \textsl{log histori} KIRI
	\item Perangkat lunak dapat melakukan data mining dari \textsl{log histori} KIRI
\end{itemize}
\subsection{Manfaat}
\begin{itemize}
	\item Agar \textsl{data log histori} KIRI dapat diolah dan didapat informasi yang menarik dan bermakna
	\item Mendapatkan sebuah perangkat lunak yang dapat melakukan \textsl{data mining} dari \textsl{log histori} KIRI
\end{itemize}

\section{Ruang Lingkup Masalah}
Penelitian \textsl{data mining} yang diatas akan ditentukan batasan masalah yang diteliti berupa : 
\begin{itemize}
	\item Penelitian ini dibatasi hanya pada permasalahan pada penerapan \textsl{data mining} pada \textsl{data log} KIRI
	\item Data log yang merupakan masukan akan dibatasi sebanyak 10000 buah data
\end{itemize}

\section{Metode Penelitian dan Teknik Pengumpulan Data}
\subsection{Metode Penelitian}
Berikut adalah Metode Penelitian yang digunakan : 
	\begin{itemize}
		\item Melakukan studi literatur tentang algoritma-algoritma yang berkaitan dengan pemrosesan \textsl{data mining}
		\item Melakukan penelitian \textsl{data mining} yang diterapkan pada \textsl{log} KIRI
		\item Merancang dan mengimplementasikan algoritma untuk \textsl{data mining}
		\item Mengimplementasikan pembangkit pola \textsl{data mining}
		\item Melakukan pengujian dan eksperimen
	\end{itemize}
\subsection{Teknik Pengumpulan Data}
Berikut adalah teknik pengumpulan data yang digunakan : 
\begin{itemize}
	\item Mengunduh dari \textsl{data log history} KIRI
\end{itemize}

\section{Sistematika Pembahasan}