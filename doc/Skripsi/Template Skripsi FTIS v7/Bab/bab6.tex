\chapter{Penutup}

\section{Kesimpulan}

Kesimpulan yang dapat diambil dari penelitian \textsl{data mining log} histori KIRI ini adalah 

Salah satu cara untuk memperoleh pola yang menarik dan bermakna, diperlukan pengolahan data dengan cara membuat klasifikasi dari data yang dihasilkan sesuai dengan tujuan yang ingin dicari. Pada penelitian ini dilakukan klasifikasi tujuan dari user apakah mereka ingin menuju Bandung atau keluar Bandung atau masih berada di area yang sama sehingga diperoleh pergerakan user KIRI di Bandung.

Pembuatan perangkat lunak untuk melakukan \textsl{data mining} pada \textsl{log} histori KIRI dapat dilakukan.

Pola yang diperoleh dari data \textsl{log} histori KIRI adalah user lebih sering menuju luar Bandung pada bulan Febuari 2014.

\section{Saran}

Untuk pengembangan aplikasi \textsl{data mining log} histori KIRI lebih lanjut, dapat dilakukan dengan cara menggunakan klasifikasi yang lebih baik dan detail serta diberi batas dengan jarak tertentu merupakan bagian luar dari Bandung sehingga \textsl{record} tersebut tidak dimasukan ke dalam proses \textsl{data mining}. Perbedaannya dengan manggunakan klasifikasi yang lebih detail adalah hasil dari \textsl{decision tree} mungkin akan lebih besar namun lebih memiliki makna dan dapat menghasilkan nilai \textsl{confident} yang lebih besar.