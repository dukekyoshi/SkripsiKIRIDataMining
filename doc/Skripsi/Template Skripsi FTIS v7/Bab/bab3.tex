\chapter{Analisa}

Pada bab ini, akan dilakukan analisa terhadap data yang akan diproses menggunakan \textsl{data mining} dan perangkat lunak yang akan dibuat untuk melakukan proses data tersebut.

\section{Analisis Data}
Pada bab ini, akan dilakukan analisa \textsl{preprocessing data} yang meliputi \textsl{data selection} dan \textsl{data transformation}. Tahap \textsl{data cleaning} dan \textsl{data integration} dapat dilewat karena hingga saat ini, database yang digunakan belum ditemukan \textsl{noise} atau \textsl{missing value}, dan hanya satu tabel (yaitu tabel \textsl{statistics}) pada database KIRI yang digunakan pada penelitian ini.

\subsection{\textsl{Data Selection}}
Pada penelitian ini, akan dilakukan proses \textsl{data mining} mengenai lokasi keberangkatan dan tujuan dari seorang user yang menggunakan aplikasi KIRI. Oleh karena itu, pada atribut \textsl{action}, nilai yang akan dipilih hanya \textsl{FINDROUTE}. Karena seluruh \textsl{action} bernilai satu jenis yaitu \textsl{FINDROUTE}, maka atribut tersebut dapat dihilangkan. Selain itu, atribut logId dan APIKey tidak akan dimasukan ke dalam proses karena tidak memiliki hubungan dengan lokasi keberangkatan dan tujuan dari seorang user.

Dari analisis diatas, maka atribut yang dipilih untuk diproses ke dalam \textsl{data mining} adalah
\begin{itemize}
	\item \textsl{Timestamp} (UTC)
	\item \textsl{AdditionalData}
\end{itemize}

Berikut contoh data dari atribut tersebut dapat dilihat pada tabel 3.1
\begin{table}[h]
\caption{Contoh data \textsl{log} KIRI setelah \textsl{data selection}}
\begin{tabular}{|l|l|}
\hline
\textbf{Timestamp (UTC)} & \textbf{AdditionalData}                     \\ \hline
2/1/2014 0:11            & -6.8972513,107.6385574/-6.91358,107.62718/1 \\ \hline
2/1/2014,0:13            & -6.8972513,107.6385574/-6.91358,107.62718/1 \\ \hline
2/1/2014 0:16            & -6.90598,107.59714/-6.90855,107.61082/1     \\ \hline
2/1/2014 0:18            & -6.9015366,107.5414474/-6.88574,107.53816/1 \\ \hline
2/1/2014 0:25            & -6.90608,107.61530/-6.89140,107.61060/2     \\ \hline
2/1/2014 0:27            & -6.89459,107.58818/-6.89876,107.60886/2     \\ \hline
2/1/2014 0:28            & -6.89459,107.58818/-6.86031,107.61287/2     \\ \hline
\end{tabular}
\end{table}

Pada atribut \textsl{additionalData}, jika nilai atribut \textsl{action} adalah \textsl{FINDROUTE}, maka nilai \textsl{addtional data} memiliki tiga bagian yang dibatasi dengan '/'. Ketiga bagian tersebut adalah

\begin{enumerate}
	\item Nilai latitude dan longitude dari lokasi keberangkatan yang dipilih oleh user
	\item Nilai latitude dan longitude dari lokasi tujuan yang dipilih oleh user
	\item Nilai yang menunjukkan banyak jalur yang dihasilkan oleh sistem KIRI
\end{enumerate}

\subsection{\textsl{Data Transformation}}
Pada atribut yang dipilih, nilai dari atribut \textsl{timestamp} dan \textsl{additionaldata} perlu dilakukan transformasi agar program dapat membaca dan memproses data lebih cepat. 

Pada atribut \textsl{timestamp}, data akan diubah menjadi tiga atribut, yaitu:
\begin{itemize}
	\item Tanggal, atribut ini akan menunjukkan tanggal ketika user KIRI memanggil \textsl{action FINDROUTE}
	\item Hari, atribut ini akan menunjukkan hari ketika user KIRI memanggil \textsl{action FINDROUTE}
	\item Jam, atribut ini akan menunjukkan jam ketika user KIRI memanggil \textsl{action FINDROUTE}
\end{itemize}

Pada atribut \textsl{additionalData}, data akan diubah menjadi tiga atribut, yaitu:
\begin{itemize}
	\item Keberangkatan, atribut ini berisi nilai latitude dan longitude dari lokasi keberangkatan yang dipilih oleh user
	\item Tujuan, atribut ini berisi Nilai latitude dan longitude dari lokasi tujuan yang dipilih oleh user
	\item Jalur, atribut ini berisi Nilai yang menunjukkan banyak jalur yang dihasilkan oleh sistem KIRI
\end{itemize}

Dari analisis diatas, banyak atribut dari tabel \textsl{statistics} akan menjadi enam, yaitu:
\begin{itemize}
	\item Tanggal
	\item Hari
	\item Jam
	\item Keberangkatan
	\item Tujuan
	\item Jalur
\end{itemize}

Contoh hasil data transformasi jika input merupakan data dari tabel 3.1 dapat dilihat pada tabel 3.2.

\begin{table}[h]
\caption{Contoh hasil data transformasi}
\begin{tabular}{|l|l|l|l|l|l|}
\hline
\textbf{Tanggal} & \textbf{Hari} & \textbf{Jam} & \textbf{Keberangkatan} & \textbf{Tujuan}    & \textbf{Jalur} \\ \hline
2/1/2014         & Sabtu         & 0:11         & -6.8972513,107.6185574 & -6.91358,107.62718 & 1              \\ \hline
2/1/2014         & Sabtu         & 0:13         & -6.8972513,107.6385574 & -6.91358,107.62718 & 1              \\ \hline
2/1/2014         & Sabtu         & 0:16         & -6.90598,107.59714     & -6.90855,107.61082 & 1              \\ \hline
2/1/2014         & Sabtu         & 0:18         & -6.9015366,107.5414474 & -6.88574,107.53816 & 1              \\ \hline
2/1/2014         & Sabtu         & 0:25         & -6.90608,107.61530     & -6.89140,107.61060 & 2              \\ \hline
2/1/2014         & Sabtu         & 0:27         & -6.89459,107.58818     & -6.89876,107.60886 & 2              \\ \hline
2/1/2014         & Sabtu         & 0:28         & -6.89459,107.58818     & -6.86031,107.61287 & 2              \\ \hline
\end{tabular}
\end{table} 

\section{Analisis Perangkat Lunak}






