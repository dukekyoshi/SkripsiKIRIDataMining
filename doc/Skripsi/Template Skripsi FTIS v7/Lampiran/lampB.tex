\chapter{The Source Code}
\label{app:B}

%selalu gunakan single spacing untuk source code !!!!!
\singlespacing 
% language: bahasa dari kode program
% terdapat beberapa pilihan : Java, C, C++, PHP, Matlab, R, dll
%
% basicstyle : ukuran font untuk kode program
% terdapat beberapa pilihan : tiny, scriptsize, footnotesize, dll
%
% caption : nama yang akan ditampilkan di dokumen akhir, lihat contoh
\begin{lstlisting}[language=Java,basicstyle=\tiny,caption=MyFurSet.java]

import java.util.ArrayList;
import java.util.Collections;
import java.util.HashSet;

/**
 *
 * @author Lionov
 */

//class for set of vertices close to furthest edge
public class MyFurSet {
    protected int id;                                  //id of the set
    protected MyEdge FurthestEdge;                     //the furthest edge
    protected HashSet<MyVertex> set;                   //set of vertices close to furthest edge
    protected ArrayList<ArrayList<Integer>> ordered;   //list of all vertices in the set for each trajectory
    protected ArrayList<Integer> closeID;              //store the ID of all vertices
    protected ArrayList<Double> closeDist;             //store the distance of all vertices
    protected int totaltrj;                            //total trajectories in the set

    /**
     * Constructor
     * @param id : id of the set
     * @param totaltrj : total number of trajectories in the set
     * @param FurthestEdge : the furthest edge
     */
    public MyFurSet(int id,int totaltrj,MyEdge FurthestEdge) {
        this.id = id;
        this.totaltrj = totaltrj;
        this.FurthestEdge = FurthestEdge;
        set = new HashSet<MyVertex>();
        ordered = new ArrayList<ArrayList<Integer>>();
        for (int i=0;i<totaltrj;i++) ordered.add(new ArrayList<Integer>());
        closeID = new ArrayList<Integer>(totaltrj);
        closeDist = new ArrayList<Double>(totaltrj);
        for (int i = 0;i <totaltrj;i++) {
            closeID.add(-1);
            closeDist.add(Double.MAX_VALUE);
        }
    }

    /**
     * set a vertex into the set
     * @param v : vertex to be added to the set
     */
    public void add(MyVertex v) {
        set.add(v);
    }

    /**
     * check whether vertex v is a member of the set
     * @param v : vertex to be checked
     * @return true if v is a member of the set, false otherwise
     */
    public boolean contains(MyVertex v) {
        return this.set.contains(v);
    }
}
\end{lstlisting}