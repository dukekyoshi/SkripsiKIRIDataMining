\documentclass[a4paper,twoside]{article}
\usepackage[T1]{fontenc}
\usepackage[bahasa]{babel}
\usepackage{graphicx}
\usepackage{graphics}
\usepackage{float}
\usepackage[cm]{fullpage}
\pagestyle{myheadings}
\usepackage{etoolbox}
\usepackage{setspace} 
\usepackage{lipsum} 
\setlength{\headsep}{30pt}
\usepackage[inner=2cm,outer=2.5cm,top=2.5cm,bottom=2cm]{geometry} %margin
% \pagestyle{empty}

\makeatletter
\renewcommand{\@maketitle} {\begin{center} {\LARGE \textbf{ \textsc{\@title}} \par} \bigskip {\large \textbf{\textsc{\@author}} }\end{center} }
\renewcommand{\thispagestyle}[1]{}
\markright{\textbf{\textsc{AIF401 \textemdash Rencana Kerja Skripsi \textemdash Sem. Ganjil 2014/2015}}}

\onehalfspacing
 
\begin{document}

\title{\@judultopik}
\author{\nama \textendash \@npm} 

%tulis nama dan NPM anda di sini:
\newcommand{\nama}{Jovan Gunawan}
\newcommand{\@npm}{2011730029}
\newcommand{\@judultopik}{\textsl{Data Mining} Histori Pencarian Rute Angkot} % Judul/topik anda
\newcommand{\jumpemb}{1} % Jumlah pembimbing, 1
\newcommand{\tanggal}{7/06/2014}
\maketitle

\pagenumbering{arabic}

\section{Deskripsi}
\textsl{Data mining} merupakan teknik untuk mengolah \textsl{data}, mencari suatu pola, kemudian dievaluasi untuk mendapatkan \textsl{knowledge}. Pola yang dihasilkan oleh \textsl{data mining} dapat digunakan untuk mencari suatu hal yang menarik dan unik. Dari pola tersebutlah, akan didapatkan suatu informasi yang sangat bermakna yang biasa disebut sebagai \textsl{knowledge}.

Pada penelitian ini, akan dilakukan penelitian data penggunaan KIRI dengan menggunakan \textsl{data mining}. Pada penelitian ini, diharapkan dapat ditemukan suatu pola yang menarik dan menghasilkan suatu ide atau gagasan yang baru. Maka dari itu, penelitian ini akan membutuhkan banyak \textsl{data} penggunaan KIRI, yang dapat diperoleh dari \textsl{log} histori KIRI.

\textsl{Log} tersebut memiliki 5 \textsl{field} untuk setiap \textsl{entry} sebagai berikut
\begin{itemize}
	\item statisticId, primary key dari entry
	\item verifier, mengidentifikasikan sumber dari pencarian ini
	\item \textsl{timestamp}, waktu ketika pengguna KIRI mencari rute angkot
	\item \textsl{type}, tipe log, untuk penelitian ini selalu berisi FINDROUTE
	\item additionalInfo, mencatat koordinat awal, koordinat akhir, dan banyak rute yang ditemukan pada pencarian ini
\end{itemize}

Dengan menggunakan \textsl{data log} histori pencarian rute angkot dari KIRI, diharapkan bisa diperoleh pola yang menarik. Pola tersebut memiliki kemungkinan untuk digunakan kembali sebagai informasi tambahan pada perangkat lunak navigasi lain seperti KIRI.

\section{Rumusan Masalah}
Dengan mengacu pada uraian diskripsi diatas, maka permasalahan yang dibahas dan diteliti oleh penulis adalah
\begin{itemize}
	\item Bagaimana cara mengolah pola yang diperoleh dari \textsl{data log histori}  KIRI agar pola menjadi menarik dan bermakna?
	\item Bagaimana membuat program untuk melakukan data mining pada data log history?
\end{itemize}

\section{Tujuan}
Penelitian ini bertujuan untuk 
\begin{itemize}
	\item Mencari pola dan informasi yang menarik dari \textsl{log histori} KIRI.
	\item Program dapat melakukan data mining dari \textsl{log histori} KIRI.
\end{itemize}

\section{Deskripsi Perangkat Lunak}
Perangkat lunak akhir yang akan dibuat memiliki fitur minimal sebagai berikut:
\begin{itemize}
	\item Pengguna dapat melakukan memasukkan database \textsl{log} KIRI.
	\item Perangkat lunak dapat melakukan \textsl{data mining} pada \textsl{log} KIRI yang sudah dimasukan.
	\item Perangkat lunak dapat menunjukkan hasil pola yang diperoleh dari \textsl{data mining}.
\end{itemize}

\section{Rencana Kerja}
Tuliskan rencana anda untuk menyelesaikan skripsi. Rencana kerja dibagi menjadi dua bagian yaitu yang akan dilakukan pada saat mengambil kuliah AIF401 Skripsi 1 dan pada saat mengambil kuliah AIF402 Skripsi 2. Perhatikan contoh berikut ini :

Rencana kerja untuk menyelesaikan skripsi ini:
\begin{itemize}
	\item Pada saat mengambil kuliah AIF401 Skripsi 1
	\begin{enumerate}
		\item Melakukan studi literatur tentang algoritma-algoritma yang berkaitan dengan pemrosesan \textsl{data mining}
		\item melakukan penelitian \textsl{data mining} yang diterapkan pada \textsl{log} KIRI
		\item Membuat dokumentasi skripsi
	\end{enumerate}
	\item Pada saat mengambil kuliah AIF401 Skripsi 2
	\begin{enumerate}
		\item Merancang dan mengimplementasikan algoritma untuk \textsl{data mining}
		\item Mengimplementasikan pembangkit pola \textsl{data mining}
		\item Melakukan pengujian dan eksperimen
		\item Membuat dokumentasi skripsi
	\end{enumerate}
\end{itemize}

\section{Isi {\it Progress Report} Skripsi 1}
Isi dari {\it Progress Report} Skripsi 1 yang akan diselesaikan paling lambat pada tanggal 1 Januari 2015 adalah :
\begin{enumerate}
	\item Algoritma/langkah-langkah untuk melakukan \textsl{data mining} dan pengambil kesimpulan dari hasil \textsl{data mining}.
	\item Hasil eksperimen penggunaan fitur-fitur Graphical User Interface pada bahasa Java
\end{enumerate}
Estimasi persentase penyelesaian skripsi sampai dengan {\it Progress Report} Skripsi 1 adalah : 85\%
\vspace{1.5cm}

\centering Bandung, \tanggal\\
\vspace{2cm} \nama \\ 
\vspace{1cm}

Menyetujui, \\
\ifdefstring{\jumpemb}{2}{
\vspace{1.5cm}
\begin{centering} Menyetujui,\\ \end{centering} \vspace{0.75cm}
\begin{minipage}[b]{0.45\linewidth}
% \centering Bandung, \makebox[0.5cm]{\hrulefill}/\makebox[0.5cm]{\hrulefill}/2013 \\
\vspace{2cm} Nama: \makebox[3cm]{\hrulefill}\\ Pembimbing Utama
\end{minipage} \hspace{0.5cm}
\begin{minipage}[b]{0.45\linewidth}
% \centering Bandung, \makebox[0.5cm]{\hrulefill}/\makebox[0.5cm]{\hrulefill}/2013\\
\vspace{2cm} Nama: \makebox[3cm]{\hrulefill}\\ Pembimbing Pendamping
\end{minipage}
\vspace{0.5cm}
}{
% \centering Bandung, \makebox[0.5cm]{\hrulefill}/\makebox[0.5cm]{\hrulefill}/2013\\
\vspace{2cm} Nama: \makebox[3cm]{\hrulefill}\\ Pembimbing Tunggal
}
`
\end{document}

